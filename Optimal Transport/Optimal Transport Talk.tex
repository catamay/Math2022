%%%%%%%%%%%%%%%%%%%%%%%%%%%%%%% beamer %%%%%%%%%%%%%%%%%%%%%%%%%%%%%%%%%%%%%%%%%%%%%%%%%
% To run - pdflatex filename.tex
%	   acroread filename.pdf
%%%%%%%%%%%%%%%%%%%%%%%%%%%%%%%%%%%%%%%%%%%%%%%%%%%%%%%%%%%%%%%%%%%%%%%%%%%%%%%%%%%%%%%%

\documentclass[compress,red,10pt]{beamer}
\mode<presentation>

\usetheme{Warsaw}
% other themes: AnnArbor, Antibes, Bergen, Berkeley, Berlin, Boadilla, boxes, CambridgeUS, Copenhagen, Darmstadt, default, Dresden, Frankfurt, Goettingen,
% Hannover, Ilmenau, JuanLesPins, Luebeck, Madrid, Maloe, Marburg, Montpellier, PaloAlto, Pittsburg, Rochester, Singapore, Szeged, classic

%\usecolortheme{lily}
% color themes: albatross, beaver, beetle, crane, default, dolphin, dov, fly, lily, orchid, rose, seagull, seahorse, sidebartab, structure, whale, wolverine

%\usefonttheme{serif}
% font themes: default, professionalfonts, serif, structurebold, structureitalicserif, structuresmallcapsserif

\hypersetup{pdfpagemode=FullScreen} % makes your presentation go automatically to full screen

% define your own colors:
\definecolor{Red}{rgb}{1,0,0}
\definecolor{Blue}{rgb}{0,0,1}
\definecolor{Green}{rgb}{0,1,0}
\definecolor{magenta}{rgb}{1,0,.6}
\definecolor{lightblue}{rgb}{0,.5,1}
\definecolor{lightpurple}{rgb}{.6,.4,1}
\definecolor{gold}{rgb}{.6,.5,0}
\definecolor{orange}{rgb}{1,0.4,0}
\definecolor{hotpink}{rgb}{1,0,0.5}
\definecolor{newcolor2}{rgb}{.5,.3,.5}
\definecolor{newcolor}{rgb}{0,.3,1}
\definecolor{newcolor3}{rgb}{1,0,.35}
\definecolor{darkgreen1}{rgb}{0, .35, 0}
\definecolor{darkgreen}{rgb}{0, .6, 0}
\definecolor{darkred}{rgb}{.75,0,0}

\xdefinecolor{olive}{cmyk}{0.64,0,0.95,0.4}
\xdefinecolor{purpleish}{cmyk}{0.75,0.75,0,0}

% can also choose different themes for the "inside" and "outside"

% \usepackage{beamerinnertheme_______}
% inner themes include circles, default, inmargin, rectangles, rounded

% \usepackage{beamerouterthemesmoothbars}
% outer themes include default, infolines, miniframes, shadow, sidebar, smoothbars, smoothtree, split, tree

\useoutertheme[subsection=false]{smoothbars}

% to have the same footer on all slides
%\setbeamertemplate{footline}[text line]{STUFF HERE!}
\setbeamertemplate{footline}[text line]{} % makes the footer EMPTY

% include packages
\usepackage{subfigure}
\usepackage{multicol}
\usepackage{amsmath}
\usepackage{epsfig}
\usepackage{graphicx}
\usepackage[all,knot]{xy}
\xyoption{arc}
\usepackage{url}
\usepackage{multimedia}
\usepackage{hyperref}
     
%%%%%%%%%%%%%%%%%%%%%%%%%%%%%%%%%%%%%%%%%%%%%%%%%%%%%%%%%%%%%%%%%%%%%%%%%%%%%%%%%%%%%%%%%%
%%%%%%%%%%%%%%%%%%%%%%%%%%%%%% Title Page Info %%%%%%%%%%%%%%%%%%%%%%%%%%%%%%%%%%%%%%%%%%%
%%%%%%%%%%%%%%%%%%%%%%%%%%%%%%%%%%%%%%%%%%%%%%%%%%%%%%%%%%%%%%%%%%%%%%%%%%%%%%%%%%%%%%%%%%

\title{Optimal Transport}
\subtitle{Preliminaries and Applications}
\author{Kirby}
\institute{Department of Mathematics\\ University of Utah}
\date{Feburary 2022}

%%%%%%%%%%%%%%%%%%%%%%%%%%%%%%%%%%%%%%%%%%%%%%%%%%%%%%%%%%%%%%%%%%%%%%%%%%%%%%%%%%%%%%%%%%
%%%%%%%%%%%%%%%%%%%%%%%%%%%%%% Begin Your Document %%%%%%%%%%%%%%%%%%%%%%%%%%%%%%%%%%%%%%%
%%%%%%%%%%%%%%%%%%%%%%%%%%%%%%%%%%%%%%%%%%%%%%%%%%%%%%%%%%%%%%%%%%%%%%%%%%%%%%%%%%%%%%%%%%

\begin{document}

%%%%%%%%%%%%%%%%%%%%%%%%%%%%%%%%%%%%%%%%%%%%%%%%%%%%%%%%%%%%%%%%%%%%%%%%%%%%%%%%%%%%%%%%%%

\frame{
	\titlepage 
}

%%%%%%%%%%%%%%%%%%%%%%%%%%%%%%%%%%%%%%%%%%%%%%%%%%%%%%%%%%%%%%%%%%%%%%%%%%%%%%%%%%%%%%%%%%

%\section[Outline]{}	% this puts the outline before EACH section automatically & will highlight the section you're about to talk about
%\frame{\tableofcontents}

%%%%%%%%%%%%%%%%%%%%%%%%%%%%%%%%%%%%%%%%%%%%%%%%%%%%%%%%%%%%%%%%%%%%%%%%%%%%%%%%%%%%%%%%%%

\section{Introduction}
\subsection{The Problem}

%%%%%%%%%%%%%%%%%%%%%%%%%%%%%%%%%%%%%%%%%%%%%%%%%%%%%%%%%%%%%%%%%%%%%%%%%%%%%%%%%%%%%%%%%%

\frame{\frametitle{Optimal Transportation}
\begin{center}
\begin{block}<+->{History}
	\vspace{.1cm}
	Originally a question of optimally moving ammunition from factory to battlefield by Gaspard Monge in 1781 during Napoleonic France.
\end{block}
\vspace{0.5cm}
\begin{block}<+->{Formulation}
	\vspace{.1cm}
	Let $X$ and $Y$ be measure spaces with measures $\mu,\nu$ respectively. In order to transport $x\in X$ to $y\in Y$, we let $\mu(X) = \nu(Y)$. 
	The optimal transport problem is minimizing the total effort (through nonnegative $c: X\times Y\to \mathbb{R}$) over $\mu$ and $\nu$.
\end{block}
\end{center}
}

%%%%%%%%%%%%%%%%%%%%%%%%%%%%%%%%%%%%%%%%%%%%%%%%%%%%%%%%%%%%%%%%%%%%%%%%%%%%%%%%%%%%%%%%%%
\frame{\frametitle{Monge Formulation}
\begin{center}
\begin{block}<+->{Transport Map}
	\vspace{.1cm}
	We call a function $T:X\to Y$ a \textit{transport map} if $T_\# \mu = \nu$ (that is, $T_\#\mu$ is the pushforward of $\nu$). Furthermore, we say $T$ transports $\mu$ to $\nu$.
\end{block}
\vspace{0.5cm}
\begin{block}<+->{Monge Formulation}
	\vspace{.1cm}
	Let
	\begin{equation}
		\mathcal{T}(\mu,\nu) = \left\{T: X\to Y \vert T_\#\mu = \nu,\,\,\text{$T$ measurable} \right\}
	\end{equation}
	Then we say that $\mathbb{M}(\mu,\nu)$ is the minimizer 
	\begin{equation}
		\mathbb{M}(\mu,\nu) = \inf_{T\in \mathcal{T}(\mu,\nu)} \int_{X}c(x,T(x))d\mu(x).
	\end{equation}
\end{block}
\end{center}
}
\frame{\frametitle{Monge Formulation Restrictions}
\begin{center}
\begin{block}<+->{Splitting Mass}
	\vspace{.1cm}
	Because we have $T: X\to Y$ with $T_\# \mu = \nu$, $T$ is bijective, so it cannot \textit{split mass}. This means for $y_1,y_2\in Y$, 
	a point $x\in X$ cannot have $T(x) = \{y_1,y_2\}$.
\end{block}
\begin{figure}
	\includegraphics[scale=0.3]{monge.jpg}
	\caption{If we let $\mu = \frac{2}{3}\delta_{x_1} + \frac{1}{3}\delta_{x_2}$ and $\nu = \frac{2}{3}\delta_{y_1} + \frac{1}{3}\delta_{y_2}$ the only valid transport map is in red.}
\end{figure}
\end{center}
}
\frame{\frametitle{Kantorovich Formulation}
\begin{center}
\begin{block}<+->{Transport Plan}
	\vspace{.1cm}
	We call a measure $\pi\in \mathcal{P}(X\times Y)$ a \textit{transport plan} if for all measurable $A\subset X$ and $B\subset Y$,
	\begin{equation}\label{eqn:kant_constraints}
		\pi(A\times Y) = \mu(A)\quad \pi(X\times B) = \nu(B).
	\end{equation}
\end{block}
\begin{block}<+->{Kantorovich Formulation}
	\vspace{.1cm}
	Let
	\begin{equation}
		\Pi(\mu,\nu) = \left\{\pi \vert \pi\in\mathcal{P}(X\times Y)\text{ constrained by \eqref{eqn:kant_constraints}} \right\}
	\end{equation}
	Then we say that $\mathbb{K}(\mu,\nu)$ is the minimizer 
	\begin{equation}
		\mathbb{K}(\mu,\nu) = \inf_{\pi\in\Pi(\mu,\nu)} \int_{X\times Y}c(x,y)d\pi(x,y).
	\end{equation}
\end{block}
\end{center}
}
\frame{\frametitle{Kantorovich Formulation as a Relaxation}
\begin{center}
	We no longer have to be restricted to not splitting mass!
\begin{figure}
	\includegraphics[scale=0.3]{kantorovich.jpg}
	\caption{Splitting mass is no problem as now $\pi(x,y)$ can be defined as necessary to provide an optimal transport scheme.}
\end{figure}
\end{center}
}
\section{Duality}
\frame{\frametitle{Shipper's Problem}
It costs $c(x_1,y_1)$ dollars for a factory to transport coal from mine $x_1$ to factory $y_1$. I tell the mine owner that I can charge them $\phi(x_1)$ dollars to pick up at mine $x_1$ and charge $\psi(y_1)$ to deliver to factory $y_1$, then, in order for the factory owner to agree to my terms, 
\[\phi(x)+\psi(y)\le c(x,y)\]
for every location $x$ and destination $y$. I can make this sum as close to $c(x,y)$ as I want, so I can maximize my profit while minimizing the factory owner's effort.
}

\frame{\frametitle{Deriving Duality}
For a measure $\pi$, we see that we have $I[\pi, \phi,\psi]$ where
\[\sup_{\phi\in C(X),\psi\in C(Y)} \left\{\int_X\phi(x)d\mu + \int_Y\psi(y)d\nu - \int_{X\times Y}(\phi(x)+\psi(y)d\pi)\right\}\]
is $0$ when $\pi\in\Pi(\mu,\nu)$ and $+\infty$ otherwise. Now, writing this as a Kantorovich Problem,
\begin{equation}
	\mathbb{K}(\mu,\nu) = \inf_{\pi\ge 0}\left\{\int_{X\times Y} c(x,y)d\pi(x,y) + \sup_{\phi\in C(X),\psi\in C(Y)} I[\pi, \phi,\psi]\right\}
\end{equation}
We can informally$\star$ exchange the $\sup$ and $\inf$ to yield 
\begin{equation}\label{eqn:minimax}
	\sup_{\phi,\psi}\left\{\int_X\phi(x)d\mu + \int_Y\psi(y)d\nu + \inf_{\pi\in\Pi}\int_{X\times Y}c(x,y) - (\phi(x)+\psi(y)d\pi))\right\}
\end{equation}
}
\frame{\frametitle{Kantorovich Duality}
Following from \eqref{eqn:minimax}, we see that
\[\inf_{\pi\in\Pi}\int_{X\times Y}c(x,y) - (\phi(x)+\psi(y)d\pi)) = \begin{cases}
	0 &\text{if $\phi(x)+\psi(y) \le c(x,y)$} \\
	+\infty &\text{otherwise.}
\end{cases}\]
This gives us the dual problem
\begin{block}<+->{Dual Problem}
	\vspace{.1cm}
	For $\mu\in \mathcal{P}(X),\nu\in\mathcal{P}(Y)$ and a nonnegative cost function $c: X\times Y\to \mathbb{R}$,
	the Kantorovich dual problem is
	\begin{equation}
		\sup_{\phi\in C(X),\psi\in C(Y)}\left\{\int_X \phi(x)d\mu(x) + \int_Y\psi(Y)d\nu(y)\right\}
	\end{equation}
	subject to $\phi(x) + \psi(y) \le c(x,y)$.
\end{block}
}
\frame{\frametitle{Discussing Informal Exchange}
There's more to the proof of Kantorovich Duality in general, but they all rely on a rigorous minimax principle.\newline
In Villani, this is done using Fenchel-Rockafeller Duality and Legendre Transforms in section 1.1.\newline
Some more important assumptions is that $X,Y$ are taken to be Polish spaces and $c$ is taken to be lower-semicontinuous in order to guarantee existence of the Kantorovich problem.\newline
This is definitely something I can talk about more in dms, so if you have specific questions, ping me in a proper channel or message me at kirby\#4923.
}
\section{Wasserstein}
\frame{\frametitle{Wasserstein Distance}
Let's start by considering Kantorovich-Monge solutions to the transport problem with cost functions $c(x,y) = d(x,y)^p$ where $d$ is the distance on $X$ and $Y$.
Let $\Omega\subset X$ and set $x_0\in X$.
\[\mathcal{P}_p(X) = \left\{\mu\in\mathcal{P}(X):\int_{\Omega}d(x,x_0)^p < +\infty\right\}\]
is the admissable class of measures $\mu$, even if $\Omega$ is unbounded.
\begin{lemma}[Kantorovich-Monge forms a metric.]\label{lem:wass}
	Let $W_p(\mu,\nu) = \left[\inf_{\pi\in\Pi(\mu,\nu)}\int_{X\times X}d(x,y)^pd\pi(x,y)\right]^{1/p}$ defined for $\mu,\nu\in\mathcal{P}_P(X)$ be the Wasserstein Distance. This forms a metric on $X$.
\end{lemma}
}
% \frame{\frametitle{Proof of Lemma \ref{lem:wass}}
% I will just be doing the triangle inequality as it's the most involved.\newline
% Let $\lambda,\mu,\nu\in\mathcal{P}_p(X)$ with $\pi_1\in\Pi(\mu,\lambda)$ and $\pi_2\in\Pi(\lambda,\nu)$ be optimal in $W_p$. So there exists $\sigma\in \mathcal{P}(X^3)$ with
% 	\begin{equation}\label{eqn:gluing}
% 		(\gamma_{1,2})_\# \sigma = \pi_1,\quad (\gamma_{2,3})_\# \sigma = \pi_2,\text{ or } (\gamma_1)_\# \sigma = \mu, \quad (\gamma_3)_\#\sigma = \nu
% 	\end{equation}
% 	so we have $(\gamma_{1,3})_\#\sigma\in\Pi(\mu,\nu)$\footnote{This is the gluing lemma in Villani 7.6, I provide without proof, but his explanation is more detailed.}. 
% 	Now, using Minkowski's Inequality and \eqref{eqn:gluing},
% 	\begin{align*}
% 		W_p(\mu,\nu) &\le \left(\int_{X\times X}d(x,z)^p\gamma_{1,3}(x,z)\right)^{1/p} \\
% 		&\le \left(\int_{X^3}d(x,y)^p\sigma(x,y,z))\right)^{1/p}  +  \left(\int_{X^3}d(y,z)^p\sigma(x,y,z))\right)^{1/p} \\
% 		&= W_p(\mu,\lambda) + W_p(\lambda,\nu).
% 	\end{align*}
% }
\frame{\frametitle{Wasserstein Topology}
Before showing that the Wasserstein Metric has a corresponding space, we see that if $x_n\to x$ in $X$ then
\(W_p(\delta_{x_n},\delta_x) = d(x_n,x)^{\inf{(1,p)}} \to 0\). In general, this looks like
\begin{theorem}[Wasserstein distances metrize weak convergence]
	Let $p\in(0,\infty)$, let $(\mu_k)_{k=1}^\infty$ be a sequence of measures in $\mathcal{P}_p(X)$ and $\mu\in \mathcal{P}(X)$, then the following are equivalent\footnote{The theorem in Villani is much stronger and gives more equivalences, but due to time, I just want to make this statement.}:
	\begin{enumerate}[(i)]
		\item $W_p(\mu_k,\mu)\to 0$
		\item $\mu_k\to\mu$ in the weak sense, meaning for $h\in C_b(X)$,
		\[\lim_{k\to\infty}\int hd\mu_k = \int hd\mu.\]
	\end{enumerate} 
\end{theorem}
Furthermore, we usually look at $p=2$, and say $\mathbb{W}_2$ is the space $\mathcal{P}_2(X)$ endowed with the metric $W_2$.
}
\section{Gradient Flows}
\frame{\frametitle{Continuity Equation}
\begin{block}<+->{Continuity Equation.}
	The linear transport equation
\end{block}
}
%%%%%%%%%%%%%%%%%%%%%%%%%%%%%%%%%%%%%%%%%%%%%%%%%%%%%%%%%%%%%%%%%%%%%%%%%%%%%%%%%%%%%%%%%%
%%%%%%%%%%%%%%%%%%%%%%%%%%%%%% End Your Document %%%%%%%%%%%%%%%%%%%%%%%%%%%%%%%%%%%%%%%%%
%%%%%%%%%%%%%%%%%%%%%%%%%%%%%%%%%%%%%%%%%%%%%%%%%%%%%%%%%%%%%%%%%%%%%%%%%%%%%%%%%%%%%%%%%%

\end{document}

